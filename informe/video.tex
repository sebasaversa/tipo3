
\section{Video}

\subsection{screen.h:}

En screen.h definimos las funciones que hacen uso de la pantalla, y definimos mem_video pasandole como valor la posicion de memoria 
donde comienza el segmento de video.

Tambi\'en definimos el struct str_memoria_video que representa los pixeles con el formato para ser mostrados por pantalla.
\begin{codesnippet}
\begin{verbatim}
/* Definicion de la pantalla */
#define VIDEO_FILS 50	
#define VIDEO_COLS 80	
#define mem_vid 0xb8000

typedef struct str_memoria_video {
    unsigned char   ascii:8;
    unsigned char   caracter:3;
    unsigned char   brillante:1;
    unsigned char   fondo:3;
    unsigned char   blink:1; 
} __attribute__((__packed__)) memoria_video;

void pintar();
void pintarTablero();
void mostrar_int(int i);
void mostrar_num(int n);
\end{verbatim}
\end{codesnippet}

\subsection{screen.c:}

\subsubsection{void pintarTablero()}
Pinta la pantalla como en la figura 8 del enunciado.
\begin{codesnippet}
\begin{verbatim}
void pintarTablero(){

    memoria_video* vd = (memoria_video*) (0xb8000);

    int f = 0;
    int c;
    while(f < VIDEO_FILS ){
        c = 0;
        while (c < VIDEO_COLS){
            if (c <= 50)
                vd->fondo = C_FG_GREEN;
            if (c >= 52)
            {
                if(f == 0 || f == 39)
                    vd->fondo = C_FG_RED;
                if( 0 < f && f < 39)
                    vd->fondo = C_FG_LIGHT_GREY;
            }
            if ( (52 < c && c < 70) && f >= 47)
                vd->fondo = C_FG_LIGHT_GREY;
			
            vd->ascii = (unsigned char) 0x0;
            vd++;	
            c++;
        }
        f++;
    }
}
\end{verbatim}
\end{codesnippet}


\subsubsection{void imprimir(char* m, memoria_video* vd)}
Recibe una cadena de caracteres y un formato, y la imprime por pantalla.
\begin{codesnippet}
\begin{verbatim}
void imprimir(char* m, memoria_video* vd){
	
    memoria_video* vdAux = vd;
    while (*m != '\0'){
        vdAux->ascii = *m;
        m++;
        vdAux++;
    }
}
\end{verbatim}
\end{codesnippet}

\subsubsection{void mostrar_int(int teclado)}
Esta funci\'on recivir\'a un entero equivalente al valor de una interrupci\'on, entre 0 y 19

\subsubsection{void mostrar_num(int teclado)}
Avanza 79 posiciones luego del inicio del segmento de video, y escribe un entero determinado por pantalla. Solo se usa con los enteros 2, 3 y 4
\begin{codesnippet}
\begin{verbatim}
void mostrar_num(int teclado){
    memoria_video* vd = (memoria_video*) (0xb8000);
    vd += 79;
    switch ( teclado)
    {
        case 2:	imprimir("1", vd);
                break;
        case 3: imprimir("2", vd);
                break;
        case 4: imprimir("3", vd);
                break;
    }
}
\end{verbatim}
\end{codesnippet}
