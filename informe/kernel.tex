
\section{Kernel.asm}

\subsection{B\'asico:}

En el kernel lo primero que hacemos es deshabilitar las interrupciones con \textbf{CLI} y cambiamos el modo de video. 
Luego, iremos habilitando y creando los diversos componentes, llamando a las funciones destinadas para \'ese fin.

Esas funciones son listadas al comienzo del c\'odigo:

\begin{codesnippet}
\begin{verbatim}
extern GDT_DESC
extern IDT_DESC 
extern idt_inicializar
extern mmu_inicializar
extern mmu_inicializar_dir_kernel
extern pintar
extern pintarTablero
extern deshabilitar_pic
extern resetear_pic
extern habilitar_pic
\end{verbatim}
\end{codesnippet}

\subsection{Preparando el modo protegido:}
Lo segundo que tenemos que hacer es habilitar A20 en el controlador del teclado para tener acceso a direcciones superiores a los $2^{20}$ bits

Luego, cargamos la \textbf{GDT} con \textit{LGDT}, pasandole el descriptor de la GDT que ya tiene el formato adecuado
\begin{codesnippet}
\begin{verbatim}
    ; Habilitar A20
    CALL habilitar_A20
    ; Cargar la GDT
    LGDT [GDT_DESC]
\end{verbatim}
\end{codesnippet}
Ahora, con la GDT cargada y A20 habilitado, podemos pasar a modo protegido seteando el bit PE de CR0 (bit 0) en 1.

Al finalizar debemos hacer un \textit{far jump} a la etiqueta de modo protegido. Es importante notar que el far jump es la unica forma de modificar el CS, operaci\'on necesaria para que no haya errores.
\begin{codesnippet}
\begin{verbatim}
; Setear el bit PE del registro CR0 (esto pasa a modo protegido)
    MOV EAX,CR0
    OR EAX,1
    MOV CR0,EAX
    ; Saltar a modo protegido
    JMP (9*0x08):modoProtegido
\end{verbatim}
\end{codesnippet}

\subsection{Modo protegido:}
Ya estamos en modo protegido. Ahora seteamos los selectores de segmentos haciendo uso de la GDT.

\begin{codesnippet}
\begin{verbatim}
    BITS 32
    modoProtegido:
        ;CODIGO
    ; Establecer selectores de segmentos
    XOR EAX, EAX
    MOV AX, 1011000b ;1011b == 11d (index de la GDT) | 0 (0 -> GDT / 1 -> LDT) |
	                                                 | 00 (NIVEL DE PRIVILEGIO)
    MOV DS, AX
    MOV ES, AX
    MOV GS, AX
    MOV SS, AX
    MOV AX, 1101000b
    MOV FS, AX
\end{verbatim}
\end{codesnippet}

Establecemos la base de la pila, llamamos a \textit{pintarTablero()} que limpia la pantalla y la pinta como en la figura 8 del enunciado. 
Luego, llamamos a idt_inicializar y, una vez inicializada, 
le pasamos el descriptor de la \textbf{IDT} a \textit{LIDT}

Por ultimo pintamos 

\begin{codesnippet}
\begin{verbatim}
    ; Establecer la base de la pila
    MOV ESP, 0x27000
    MOV EBP, ESP
    CALL pintarTablero
    CALL idt_inicializar
    LIDT [IDT_DESC]
\end{verbatim}
\end{codesnippet}

\subsection{Paginaci\'on:}

Primero inicializamos las p\'aginas con \textit{identity mapping} y cargo en CR3 la direccion del \textit{page directory}
Luego modifico CR0 para habilitar paginaci\'on

\begin{codesnippet}
\begin{verbatim}
	CALL mmu_inicializar_dir_kernel
	MOV EAX, 0x27000
	MOV CR3, EAX ;cargo en CR3 la direccion del page directory
	MOV EAX, CR0
    OR EAX, 0x80000000 ;habilitamos paginacion
	MOV CR0, EAX
\end{verbatim}
\end{codesnippet}

Algo
\begin{codesnippet}
\begin{verbatim}
    ; Imprimir mensaje de bienvenida

    ; Inicializar pantalla
    
    ; Inicializar el manejador de memoria
\end{verbatim}
\end{codesnippet}

Algo
directorio de paginas de tanques?
\begin{codesnippet}
\begin{verbatim}
    ; Inicializar el directorio de paginas
    CALL mmu_inicializar
    
    CALL deshabilitar_pic
    CALL resetear_pic
    CALL habilitar_pic
    STI
\end{verbatim}
\end{codesnippet}

\textbf{RESTO}
