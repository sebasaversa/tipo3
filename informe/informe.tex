\documentclass[a4paper]{article}
\usepackage[spanish]{babel}
\usepackage[utf8]{inputenc}
\usepackage{charter}   % tipografia
\usepackage{graphicx}
%\usepackage{makeidx}

%\usepackage{float}
%\usepackage{amsmath, amsthm, amssymb}
%\usepackage{amsfonts}
%\usepackage{sectsty}
%\usepackage{charter}
%\usepackage{wrapfig}
%\usepackage{listings}
%\lstset{language=C}


\input{codesnippet}
\input{page.layout}
% \setcounter{secnumdepth}{2}
\usepackage{underscore}
\usepackage{caratula}
\usepackage{url}


% ******************************************************** %
%              TEMPLATE DE INFORME ORGA2 v0.1              %
% ******************************************************** %
% ******************************************************** %
%                                                          %
% ALGUNOS PAQUETES REQUERIDOS (EN UBUNTU):                 %
% ========================================
%                                                          %
% texlive-latex-base                                       %
% texlive-latex-recommended                                %
% texlive-fonts-recommended                                %
% texlive-latex-extra?                                     %
% texlive-lang-spanish (en ubuntu 13.10)                   %
% ******************************************************** %



\begin{document}


\thispagestyle{empty}
\materia{Organización del Computador II}
\submateria{Primer Cuatrimestre de 2014}
\titulo{Trabajo Práctico III}
\subtitulo{TronTank}
\integrante{Tomas Shaurli}{671/10}{tshaurli@gmail.com}
\integrante{Sebastian Aversa}{379/11}{sebastianaversa@gmail.com}
\integrante{Fernando Gabriel Otero}{424/11}{fergabot@gmail.com}

\maketitle
\newpage

\thispagestyle{empty}
\vfill
\begin{abstract}
En el presente trabajo se realiza la configuraci\'on de un kernel de sistema operativo de 32bits con sus elementos indispensables (GDT, IDT, TSS, paginaci\'on, etc) y el manejo de tareas.
\end{abstract}
\newpage
\thispagestyle{empty}
\vspace{3cm}
\tableofcontents
\newpage


\normalsize
\newpage

\section{Objetivos generales}

Este trabajo pr\'actico consisti\'o en un conjunto de ejercicios para el parendizaje de los conceptos de System Programming
\\
Se implement\'o para ello un sistema m\'inimo en base a los archivos aportados por la c\'atedra, el cual ser\'a capaz de manejar exactamente 8 tareas a nivel de usuario.
El sistema ser\'a capaz de capturar cualquier problema que puedan generar las tareas y tomar las acciones necesarias para quitar a la tarea del sistema.
Los ejercicios de este trabajo pr\'actico proponen utilizar los mecanismos que posee el procesador para la programaci\'on desde el punto de vista del sistema operativo 
enfocados en dos aspectos: el sistema de protecci\'on y la ejecuci\'on concurrente de tareas.

%\newpage
%\input{introduccion}
\newpage

\section{Kernel.asm}

\subsection{B\'asico:}

En el kernel lo primero que hacemos es deshabilitar las interrupciones con \textbf{CLI} y cambiamos el modo de video. 
Luego, iremos habilitando y creando los diversos componentes, llamando a las funciones destinadas para \'ese fin.

Esas funciones son listadas al comienzo del c\'odigo:

\begin{codesnippet}
\begin{verbatim}
extern GDT_DESC
extern IDT_DESC 
extern idt_inicializar
extern mmu_inicializar
extern mmu_inicializar_dir_kernel
extern pintar
extern pintarTablero
extern deshabilitar_pic
extern resetear_pic
extern habilitar_pic
\end{verbatim}
\end{codesnippet}

\subsection{Preparando el modo protegido:}
Lo segundo que tenemos que hacer es habilitar A20 en el controlador del teclado para tener acceso a direcciones superiores a los $2^{20}$ bits

Luego, cargamos la \textbf{GDT} con \textit{LGDT}, pasandole el descriptor de la GDT que ya tiene el formato adecuado
\begin{codesnippet}
\begin{verbatim}
    ; Habilitar A20
    CALL habilitar_A20
    ; Cargar la GDT
    LGDT [GDT_DESC]
\end{verbatim}
\end{codesnippet}
Ahora, con la GDT cargada y A20 habilitado, podemos pasar a modo protegido seteando el bit PE de CR0 (bit 0) en 1.

Al finalizar debemos hacer un \textit{far jump} a la etiqueta de modo protegido. Es importante notar que el far jump es la unica forma de modificar el CS, operaci\'on necesaria para que no haya errores.
\begin{codesnippet}
\begin{verbatim}
; Setear el bit PE del registro CR0 (esto pasa a modo protegido)
    MOV EAX,CR0
    OR EAX,1
    MOV CR0,EAX
    ; Saltar a modo protegido
    JMP (9*0x08):modoProtegido
\end{verbatim}
\end{codesnippet}

\subsection{Modo protegido:}
Ya estamos en modo protegido. Ahora seteamos los selectores de segmentos haciendo uso de la GDT.

\begin{codesnippet}
\begin{verbatim}
    BITS 32
    modoProtegido:
        ;CODIGO
    ; Establecer selectores de segmentos
    XOR EAX, EAX
    MOV AX, 1011000b ;1011b == 11d (index de la GDT) | 0 (0 -> GDT / 1 -> LDT) |
	                                                 | 00 (NIVEL DE PRIVILEGIO)
    MOV DS, AX
    MOV ES, AX
    MOV GS, AX
    MOV SS, AX
    MOV AX, 1101000b
    MOV FS, AX
\end{verbatim}
\end{codesnippet}

Establecemos la base de la pila, llamamos a \textit{pintarTablero()} que limpia la pantalla y la pinta como en la figura 8 del enunciado. 
Luego, llamamos a idt_inicializar y, una vez inicializada, 
le pasamos el descriptor de la \textbf{IDT} a \textit{LIDT}

Por ultimo pintamos 

\begin{codesnippet}
\begin{verbatim}
    ; Establecer la base de la pila
    MOV ESP, 0x27000
    MOV EBP, ESP
    CALL pintarTablero
    CALL idt_inicializar
    LIDT [IDT_DESC]
\end{verbatim}
\end{codesnippet}

\subsection{Paginaci\'on:}

mmu_inicializar_dir_kernel
MOV CR3, EAX ;cargo en CR3 la direccion del page directory

MOV EAX, CR0
OR EAX, 0x80000000 ;habilitamos paginacion
MOV CR0, EAX

\begin{codesnippet}
\begin{verbatim}
	CALL mmu_inicializar_dir_kernel
	MOV EAX, 0x27000
	MOV CR3, EAX ;cargo en CR3 la direccion del page directory
	MOV EAX, CR0
    OR EAX, 0x80000000 ;habilitamos paginacion
	MOV CR0, EAX
\end{verbatim}
\end{codesnippet}
\begin{codesnippet}
\begin{verbatim}
    ; Imprimir mensaje de bienvenida

    ; Inicializar pantalla
    
    ; Inicializar el manejador de memoria
\end{verbatim}
\end{codesnippet}
\begin{codesnippet}
\begin{verbatim}
    ; Inicializar el directorio de paginas
    CALL mmu_inicializar
    
    CALL deshabilitar_pic
    CALL resetear_pic
    CALL habilitar_pic
    STI
\end{verbatim}
\end{codesnippet}

\textbf{RESTO}

\newpage

\section{GDT}
\subsection{gdt.h}
En el archivo gdt.h s\'olo se modifico el define GDT_COUNT, d\'andole 20 por valor ya que es la cantidad de entradas que tiene la GDT

\subsection{gdt.c}
En gdt.c se tom\'o la extructura dada por la c\'atedra y se agregaron las entradas correspondientes.

\begin{itemize}
	\item La entrada nula, en la posicion 0
	
	\item Dos descriptores de segmentos de datos, uno de prioridad 0 y otro de prioridad 3

	\item Dos descriptores de segmentos de c\'odigo, uno de prioridad 0 y otro de prioridad 3
	
	\item El descriptor del segmento de video
	
	\item Tres descriptores de TSS, el primero para la tarea \textit{Idle}
\end{itemize}

\newpage

\section{Video}

\subsection{screen.h:}

En screen.h definimos las funciones que hacen uso de la pantalla, y definimos mem_video pasandole como valor la posicion de memoria 
donde comienza el segmento de video.

Tambi\'en definimos el struct str_memoria_video que representa los pixeles con el formato para ser mostrados por pantalla.
\begin{codesnippet}
\begin{verbatim}
/* Definicion de la pantalla */
#define VIDEO_FILS 50	
#define VIDEO_COLS 80	
#define mem_vid 0xb8000

typedef struct str_memoria_video {
    unsigned char   ascii:8;
    unsigned char   caracter:3;
    unsigned char   brillante:1;
    unsigned char   fondo:3;
    unsigned char   blink:1; 
} __attribute__((__packed__)) memoria_video;

void pintar();
void pintarTablero();
void mostrar_int(int i);
void mostrar_num(int n);
\end{verbatim}
\end{codesnippet}

\subsection{screen.c:}

\subsubsection{void pintarTablero()}
Pinta la pantalla como en la figura 8 del enunciado.
\begin{codesnippet}
\begin{verbatim}
void pintarTablero(){

    memoria_video* vd = (memoria_video*) (0xb8000);

    int f = 0;
    int c;
    while(f < VIDEO_FILS ){
        c = 0;
        while (c < VIDEO_COLS){
            if (c <= 50)
                vd->fondo = C_FG_GREEN;
            if (c >= 52)
            {
                if(f == 0 || f == 39)
                    vd->fondo = C_FG_RED;
                if( 0 < f && f < 39)
                    vd->fondo = C_FG_LIGHT_GREY;
            }
            if ( (52 < c && c < 70) && f >= 47)
                vd->fondo = C_FG_LIGHT_GREY;
			
            vd->ascii = (unsigned char) 0x0;
            vd++;	
            c++;
        }
        f++;
    }
}
\end{verbatim}
\end{codesnippet}


\subsubsection{void imprimir(char* m, memoria_video* vd)}
Recibe una cadena de caracteres y un formato, y la imprime por pantalla.
\begin{codesnippet}
\begin{verbatim}
void imprimir(char* m, memoria_video* vd){
	
    memoria_video* vdAux = vd;
    while (*m != '\0'){
        vdAux->ascii = *m;
        m++;
        vdAux++;
    }
}
\end{verbatim}
\end{codesnippet}

\subsubsection{void mostrar_int(int teclado)}
Esta funci\'on recivir\'a un entero equivalente al valor de una interrupci\'on, entre 0 y 19

\subsubsection{void mostrar_num(int teclado)}
Avanza 79 posiciones luego del inicio del segmento de video, y escribe un entero determinado por pantalla. Solo se usa con los enteros 2, 3 y 4
\begin{codesnippet}
\begin{verbatim}
void mostrar_num(int teclado){
    memoria_video* vd = (memoria_video*) (0xb8000);
    vd += 79;
    switch ( teclado)
    {
        case 2:	imprimir("1", vd);
                break;
        case 3: imprimir("2", vd);
                break;
        case 4: imprimir("3", vd);
                break;
    }
}
\end{verbatim}
\end{codesnippet}

\newpage

\section{Interrupciones de sistema}
\subsection{idt.c}
La idt se genera utilizando la macro dada por la catedra (completada por nosotros).
\begin{codesnippet}
\begin{verbatim}
#define IDT_ENTRY(numero)                                                                                        \
    idt[numero].offset_0_15 = (unsigned short) ((unsigned int)(&_isr ## numero) & (unsigned int) 0xFFFF);        \
    idt[numero].segsel = (unsigned short) (GDT_IDX_CODE_DESC_1*8);                                                                  \
    idt[numero].attr = (unsigned short) 0x8E00;                                                                  \
    idt[numero].offset_16_31 = (unsigned short) ((unsigned int)(&_isr ## numero) >> 16 & (unsigned int) 0xFFFF);
\end{verbatim}
\end{codesnippet}

Esta macro se usa dada la similitud entre todas las interrupciones manejadas en este archivo. De hecho, en idt.c la unica diferencia 
entre dos interrupciones es su n\'umero
\subsection{screen.c}
Aqu\'i se encuentra la funci\'on \textit{mostrar_int(int error)} con los mensajes correspondentes para cada interrupcion
\subsection{isr.asm}

En el archivo isr.asm completamos la informaci\'on de las rutinas de atencion de interrupciones

\newpage

\section{MMU}
\subsection{A modificar}
\subsubsection{codigo util}
\textbf{cli} \textit{cli}

\begin{codesnippet}
\begin{verbatim}
    ; Habilitar A20
    CALL habilitar_A20
    ; Cargar la GDT
    LGDT [GDT_DESC]
\end{verbatim}
\end{codesnippet}

\newpage

\section{Interrupciones de reloj y teclado}
\subsection{Modificaciones:}

En el kernel lo primero que hacemos es deshabilitar las interrupciones con \textbf{cli} y cambiamos el modo de video. 
Luego, iremos habilitando y creando los diversos componentes, llamando a las funciones destinadas para \'ese fin.
\\
Esas funciones son listadas al comienzo del c\'odigo:

\begin{codesnippet}
\begin{verbatim}
extern GDT_DESC
extern IDT_DESC 
extern idt_inicializar
extern mmu_inicializar
extern mmu_inicializar_dir_kernel
extern pintar
extern pintarTablero
extern deshabilitar_pic
extern resetear_pic
extern habilitar_pic
\end{verbatim}
\end{codesnippet}

\begin{codesnippet}
\begin{verbatim}
    ; Habilitar A20
    CALL habilitar_A20
    ; Cargar la GDT
    LGDT [GDT_DESC]
\end{verbatim}
\end{codesnippet}

\begin{codesnippet}
\begin{verbatim}
; Setear el bit PE del registro CR0 (esto pasa a modo protegido)
    MOV EAX,CR0
    OR EAX,1
    MOV CR0,EAX
    ; Saltar a modo protegido
    JMP (9*0x08):modoProtegido

    BITS 32
\end{verbatim}
\end{codesnippet}

\begin{codesnippet}
\begin{verbatim}
    modoProtegido:
        ;CODIGO
        
    ; Establecer selectores de segmentos
    XOR EAX, EAX
    MOV AX, 1011000b ;1011b == 11d (index de la GDT) | 0 (0 -> GDT / 1 -> LDT) | 00 (NIVEL DE PRIVILEGIO)
    MOV DS, AX
    MOV ES, AX
    MOV GS, AX
    MOV SS, AX
    MOV AX, 1101000b
    MOV FS, AX
    ; Establecer la base de la pila
    MOV ESP, 0x27000
    MOV EBP, ESP
    CALL pintar
    CALL idt_inicializar
	LIDT [IDT_DESC]

    xor eax, eax
    xor edi, edi
    div edi	
	CALL pintarTablero
\end{verbatim}
\end{codesnippet}

\begin{codesnippet}
\begin{verbatim}
	CALL mmu_inicializar_dir_kernel
	MOV EAX, 0x27000
	MOV CR3, EAX ;cargo en CR3 la direccion del page directory
	MOV EAX, CR0
    OR EAX, 0x80000000 ;habilitamos paginacion
	MOV CR0, EAX
    ; Imprimir mensaje de bienvenida

    ; Inicializar pantalla
    
    ; Inicializar el manejador de memoria
    
    ; Inicializar el directorio de paginas
    CALL mmu_inicializar
    
    CALL deshabilitar_pic
    CALL resetear_pic
    CALL habilitar_pic
    STI
\end{verbatim}
\end{codesnippet}

\newpage
\input{tss}
\newpage
\input{scheduler}
\newpage
%\section{Conclusiones y trabajo futuro}
%asdf4


\end{document}

