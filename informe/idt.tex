
\section{Interrupciones de sistema}
\subsection{idt.c}
La idt se genera utilizando la macro dada por la catedra (completada por nosotros).
\begin{codesnippet}
\begin{verbatim}
#define IDT_ENTRY(numero)                                                                                        \
    idt[numero].offset_0_15 = (unsigned short) ((unsigned int)(&_isr ## numero) & (unsigned int) 0xFFFF);        \
    idt[numero].segsel = (unsigned short) (GDT_IDX_CODE_DESC_1*8);                                                                  \
    idt[numero].attr = (unsigned short) 0x8E00;                                                                  \
    idt[numero].offset_16_31 = (unsigned short) ((unsigned int)(&_isr ## numero) >> 16 & (unsigned int) 0xFFFF);
\end{verbatim}
\end{codesnippet}

Esta macro se usa dada la similitud entre todas las interrupciones manejadas en este archivo. De hecho, en idt.c la unica diferencia 
entre dos interrupciones es su n\'umero
\subsection{screen.c}
Aqu\'i se encuentra la funci\'on \textit{mostrar_int(int error)} con los mensajes correspondentes para cada interrupcion
\subsection{isr.asm}

En el archivo isr.asm completamos la informaci\'on de las rutinas de atencion de interrupciones
